\documentclass[12pt]{article}

\usepackage{fontspec}
\usepackage{geometry}
\usepackage{lastpage}
\usepackage{fancyhdr}
\usepackage{hyperref}

\geometry{top=1in, bottom=1in, left=1in, right=1in, marginparsep=4pt, marginparwidth=1in}

\renewcommand{\headrulewidth}{0pt}
\pagestyle{fancyplain}
\fancyhf{}
\cfoot{\thepage\ of \pageref{LastPage}}

\setlength{\parindent}{0pt}
\setlength{\parskip}{0pt}

\usepackage{marginnote} % For margin years
\newcommand{\years}[1]{\marginnote{\scriptsize #1}} % New command for including margin years
\renewcommand*{\raggedleftmarginnote}{}
\setlength{\marginparsep}{-16pt} % Slightly increase the distance of the margin years from the content
\reversemarginpar

\setromanfont [Ligatures={Common}, Numbers={OldStyle}, Variant=01,
 BoldFont={LinLibertine_RB.otf},
 ItalicFont={LinLibertine_RI.otf},
 BoldItalicFont={LinLibertine_RBI.otf}
 ]{LinLibertine_R.otf}
%\setromanfont [Ligatures={Common}, Numbers={OldStyle}]{Hoefler Text}

%\usepackage[xetex, bookmarks, pdftitle={Taylor Arnold CV},pdfauthor={Taylor Arnold}]{hyperref}
%\hypersetup{linkcolor=blue,citecolor=blue,filecolor=black,urlcolor=MidnightBlue}

\usepackage{xunicode} % Allows generation of unicode characters from accented glyphs
\defaultfontfeatures{Mapping=tex-text}

\begin{document}

\begin{center}
{\bf Linear Models: STAT 312a / STAT 612a} \\
Fall 2015 \quad Monday, Wednesdays 11:35 - 12:50 \quad 60 Sachem Street (Watson Center), Rm A60
\end{center}

\bigskip

\noindent
\begin{tabular}{ l l }
{\bf Instructor:} &  {\bf Taylor Arnold} \\
E-mail: & \href{mailto:taylor.arnold@yale.edu}{taylor.arnold@yale.edu} \\
Office: & 24 Hillhouse, Rm 203 \\
Office Hours: & Wednesdays 13:30-15:00 \\
Teaching Assistant: & Jason Klusowski \\
TA E-mail: & \href{mailto:jason.klusowski@yale.edu}{jason.klusowski@yale.edu} \\
TA Hours: & TBD
\end{tabular}

\vspace{1cm}

{\bf Course Description:} \\
The geometry of least squares; distribution theory for normal errors; regression, analysis of variance, and designed experiments; numerical algorithms, with particular reference to the R statistical language.

\vspace{0.5cm}

{\bf Grading:}
\begin{itemize}\setlength\itemsep{0em}
\item 70\% Problem Sets
\item 15\% Mid-Term I (2015-10-12)
\item 15\% Mid-Term II (2015-11-18)
\end{itemize}

\vspace{0.5cm}

{\bf Suggested Prerequisites:}
\begin{itemize}\setlength\itemsep{0em}
\item Linear Algebra at the level of MATH 222
\item Statistical theory at the level of STAT 242
\item Some familiarity with a statistical software or programming language, preferably R
\end{itemize}

\vspace{0.5cm}

{\bf Suggested References:}
\begin{itemize}\setlength\itemsep{0em}
\item Rao, Calyampudi R., et al. {\it Linear Models and Generalizations}. Springer New York, 2008.
\item Hayashi, Fumio. {\it Econometrics}. Princeton University Press, 2000.
\item Golub, Gene H., and Charles F. Van Loan. {\it Matrix computations}. Vol. 3. JHU Press, 2012.
\item Bühlmann, Peter, and Sara Van De Geer. {\it Statistics for high-dimensional data: methods, theory and applications}. Springer Science \& Business Media, 2011.
\end{itemize}

\vspace{0.5cm}

{\bf Problem Sets:} \\
Problem sets are assigned roughly once every two weeks;
this yields a total of 7 sets.
You may discuss problem sets with other students, but must write up your
own solutions. This means that you should have no need to look at other
student's final written solutions.

\bigskip

Tentative due dates for problem sets: 09-14, 09-28, 10-05, 10-19, 11-02,
11-09 and 12-16.

% \newpage

% {\bf Tentative Schedule:}
% \begin{itemize}\setlength\itemsep{0em}
% \item 2015-09-02: \, Course introduction; simple linear model assumptions and MLEs (RT 2.1-2.7)
% \item 2015-09-07: \, Hypothesis tests; best linear unbiased estimators (RT 2.8-2.10)
% \item 2015-09-09: \, Multivariate linear regression; normal equations and OLS (FH 1.1-1.3)
% \item 2015-09-14: \, Finite sample distribution theory for multivariate regression (FH 1.4)
% \item 2015-09-16: \, Distribution theory continued; Multivariate MLE (FH 1.5)
% \item 2015-09-21: \, Geometry of regression; best linear unbiased estimators (RT 3.3-3.4)
% \item 2015-09-23: \, Analysis of variance (RT 3.9)
% \item 2015-09-28: \, Large sample theory for multivariate regression (FH 2.2)
% \item 2015-09-30: \, Large sample theory for multivariate regression, cont. (FH 2.3)
% \item 2015-10-05: \, Large sample theory for multivariate regression, cont. (FH 2.4)
% \item 2015-10-07: \, Weighted least squares and heteroskedasticity (FH 1.6, 2.9)
% \item 2015-10-12: \, MIDTERM I
% \item 2015-10-14: \, Solving full rank least squares (GV 5.2, 5.3)
% \item 2015-10-19: \, Sensitivity of the least squares solution (GV 5.3)
% \item 2015-10-21: \, FALL BREAK
% \item 2015-10-26: \, Iterative methods for least squares (GV 10.2, 10.3)
% \item 2015-10-28: \, Principal component regression; Ridge regression (RT 3.14)
% \item 2015-11-02: \, Shrinkage estimators (RT 3.14)
% \item 2015-11-04: \, Lasso regression; geometric formulation and basic theory (BV 2.1-2.3, 6.1-6.3)
% \item 2015-11-09: \, Calculating the lasso; LARS path algorithm (BV 2.12)
% \item 2015-11-11: \, Lasso theory (BV 6.1-6.7)
% \item 2015-11-16: \, Prediction measures; model selection criteria (RT 7.2, 7.9)
% \item 2015-11-18: \, MIDTERM II
% \item 2015-11-23: \, THANKSGIVING BREAK
% \item 2015-11-25: \, THANKSGIVING BREAK
% \item 2015-11-30: \, Bayesian regression
% \item 2015-12-02: \, Least absolute deviation (RT 9.2)
% \item 2015-12-07: \, M-estimators (RT 9.3, 9.5)
% \item 2015-12-09: \, Statistical tests for robust estimators (RT 9.6)
% \end{itemize}

\end{document}





