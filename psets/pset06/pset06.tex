\documentclass[12pt]{article}

\usepackage{fontspec}
\usepackage{geometry}
\usepackage{lastpage}
\usepackage{fancyhdr}
\usepackage{hyperref}
\usepackage{amsmath}
\usepackage{amsthm}
\usepackage{amssymb}

\makeatletter
\newcommand{\distas}[1]{\mathbin{\overset{#1}{\kern\z@\sim}}}%

\geometry{top=1in, bottom=1in, left=1in, right=1in, marginparsep=4pt, marginparwidth=1in}

\renewcommand{\headrulewidth}{0pt}
\pagestyle{fancyplain}
\fancyhf{}
\cfoot{\thepage\ of \pageref{LastPage}}

\setlength{\parindent}{0pt}
\setlength{\parskip}{12pt}

\usepackage{marginnote} % For margin years
\newcommand{\years}[1]{\marginnote{\scriptsize #1}} % New command for including margin years
\renewcommand*{\raggedleftmarginnote}{}
\setlength{\marginparsep}{-16pt} % Slightly increase the distance of the margin years from the content
\reversemarginpar

\setromanfont [Ligatures={Common}, Numbers={OldStyle}, Variant=01,
 BoldFont={LinLibertine_RB.otf},
 ItalicFont={LinLibertine_RI.otf},
 BoldItalicFont={LinLibertine_RBI.otf}
 ]{LinLibertine_R.otf}
%\setromanfont [Ligatures={Common}, Numbers={OldStyle}]{Hoefler Text}

%\usepackage[xetex, bookmarks, pdftitle={Taylor Arnold CV},pdfauthor={Taylor Arnold}]{hyperref}
%\hypersetup{linkcolor=blue,citecolor=blue,filecolor=black,urlcolor=MidnightBlue}

\usepackage{xunicode} % Allows generation of unicode characters from accented glyphs
\defaultfontfeatures{Mapping=tex-text}

\begin{document}

\begin{center}
{\bf Problem Set 05} \\
Linear Models -- Fall 2015 \\
Due date: 2015-11-18
\end{center}

\medskip

Problems sets are due at the start of class on the due date. Please hand write
or type up and print the solutions; we will not accept e-mail solution sets except
in exceptional circumstances. You may discuss problem sets with others, but must
write up your own solutions. This means that you should have no need to look at other's
final written solutions. Many of these problems come from a variety of textbooks,
which are referenced in the problems. These are for citation purposes and not because
you will need to consult the text itself (though you may feel free to do so).

\medskip

This problem set comes courtesy of David Pollard, who put together a fantastic
set of notes several years ago for his Linear Models course which guides you
along the LARs path algorithm for solving the lasso. I have stored a copy
of these here:
\begin{quote}
\url{http://euler.stat.yale.edu/~tba3/psets/pset06/pset06_dp.pdf}
\end{quote}
Following David's convention, you must at least hand in solutions to the questions
with a $(*)$ next to them. The other questions are a bit harder and we will accept
for extra credit.

\end{document}





